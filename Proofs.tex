\documentclass[a4paper,12pt]{article}
\usepackage{amsmath, amsthm}
\usepackage{amsfonts}
\usepackage{amssymb}
\usepackage[utf8]{inputenc}
\usepackage[T1]{fontenc}
\usepackage{slashed}
\usepackage{commath}
\usepackage[style=phys, biblabel=brackets, eprint=true, sorting=none, backend=biber, maxnames=99]{biblatex}
\usepackage{graphicx}
\usepackage{xr}
\externaldocument[form]{CollectionOfFormulas}

%Bibliographies
\addbibresource{References.bib}

%Page Settings
\setlength{\textwidth}{14.5cm}
\setlength{\textheight}{22.5cm}
\setlength{\topmargin}{-5mm}%{-1.5cm}
%\setlength{\evensidemargin}{0.36cm}
\setlength{\oddsidemargin}{5mm}%{0.36cm}

%Better, emptier empty set
\renewcommand{\emptyset}{\varnothing}

\newcommand{\GA}[2][]{\ensuremath{\mathcal{G}_{#1}({#2})}}
\newcommand{\grade}[2]{\ensuremath{\langle#2\rangle_{#1}}}
\newcommand{\deriv}[2]{\ensuremath{\frac{\mathrm{d}{#1}}{\mathrm{d}{#2}}}}
\newcommand{\cev}[1]{\reflectbox{\ensuremath{\vec{\reflectbox{\ensuremath{#1}}}}}}
\newcommand{\reverse}[1]{\tilde{#1}} %The Clifford reverse
\newcommand{\reals}[1]{\ensuremath{\mathbb{R}^{#1}}}
\newcommand{\vol}[1]{\ensuremath{\mathrm{vol}(#1)}}
\newcommand{\linner}{\mathbin{\ensuremath{\scalebox{1.4}{$\lrcorner$}}}}
\newcommand{\rinner}{\mathbin{\ensuremath{\scalebox{1.4}{$\llcorner$}}}}

\newcommand{\comment}[1]{}

\providecommand{\abs}[1]{\lvert#1\rvert}
\providecommand{\norm}[1]{\left\lVert#1\right\rVert}
\providecommand{\normed}[1]{\hat{#1}}
\DeclareMathOperator{\sign}{sgn}
\DeclareMathOperator{\tr}{Tr}
\providecommand{\signX}[1]{\ensuremath{s_{#1}}}

\newcommand{\meqref}[1]{M.\eqref{form#1}}

\newtheorem{thm}{Theorem}
\newtheorem{lem}[thm]{Lemma}
\newtheorem{cor}[thm]{Corollary}

\theoremstyle{definition}
\newtheorem{defn}{Definition}

\newenvironment{minipeqn}[1][]{\begin{minipage}[#1]{.45\textwidth}\begin{equation}}{\nonumber\end{equation}\end{minipage}}

%opening
\title{Proofs for Formulas in Geometric Algebra and Geometric Calculus}

\author{}

\begin{document}

\maketitle

\section{Introduction}

This file shall contain proofs and derivations for such formulas in the main collection which are not found in published literature. If you want to submit a pull request for a formula which you derived yourself, and have not published an article exposing the proof that could be cited, this is the place to submit the derivation.

Formulae in this document will be referenced as usual, (xx), and formulae in the main document as M.(xx).
\section{Multivector directed derivatives}

Many of the following are present and proven in the literature, however, as they follow in a nice logical sequence, we present the proofs here.
We wish to remind the reader that $\signX{X} = \sign{X * \reverse{X}}$, as defined in the main text.

\paragraph{Proof of \meqref{eq:normsqrdif}:} write $\norm{X}^2 = \signX{X} X * \reverse{X}$, use the product rule (Proposition 8. in \cite{HitzerCalculus}), (1.20a) in \cite{CA2GC} and \meqref{eq:scalpdiff}.

\paragraph{Proof of \meqref{eq:normdif}:} Write $\norm{X} = \sqrt{\norm{X}^2}$ and use the chain rule with $x \mapsto \sqrt{x}$ ($x$ a real number) and \meqref{eq:normsqrdif}.

\paragraph{Proof of \meqref{eq:normeddif}:} write
\begin{equation}
A*\partial \normed{X} = A*\dot{\partial} \frac{\dot{X}}{\norm{X}} + A*\dot{\partial} X \frac{1}{\dot{\norm{X}}} = \frac{A}{\norm{X}} - X \frac{\signX{X} \frac{A * \reverse{X}}{\norm{X}}}{\norm{X}^2}
\end{equation}
and simplify, where we used the product rule and chain rule on $x \mapsto 1/x$ and $\meqref{eq:normeddif}$.
The second line follows immediately from the definition of the inverse \meqref{eq:inversedef} and the projection \meqref{eq:projdef}.

\paragraph{Proof of \meqref{eq:normtokdif}:} use the chain rule and \meqref{eq:normdif} on $\norm{X}$ composed with $x \mapsto x^k$ for real $x$.

\paragraph{Proof of \meqref{eq:evenbladedif}:} for a blade, $X_k^2 = X_k * X_k$.
Apply the product rule to get $A_k*\partial X_k^2 = 2 A_k * X_k$, and the chain rule to $(X_k^2)^{n/2}$ to get the result.
The second line follows from $X_k^{-1} = \frac{X_k}{X_k^2}$, when $X_k$ is invertible.

\paragraph{Proof of \meqref{eq:oddbladedif}:} Write $X_k^n = X_k X_k^{n-1}$, where now $n-1$ is even.
Use the product rule and the fact that $X_k^{n-1}$ commutes, and apply \meqref{eq:evenbladedif}.
The second line again follows from the definition of the projection when $X_k$ is invertible.

\paragraph{Proof of \meqref{eq:bladeexpdif}:} for $X_k^2 < 0$: use \meqref{eq:bladeexptrig} to write
\begin{equation}
A_k * \partial e^{X_k} = A_k * \partial(\cos(\norm{X_k}) + \normed{X_k} \sin(\norm{X_k})),
\end{equation}
apply the chain rule and \meqref{eq:normdif} to differentiate the sine and cosine terms, and \meqref{eq:normeddif} ($X_k$ is invertible since $X_k^2 < 0$) to differentiate $\normed{X_k}$ to get
\begin{equation}
\begin{split}
A_k * \partial e^{X_k} =& -\signX{X_k}\frac{A_k*\reverse{X_k}}{\norm{X_k}}\sin(\norm{X_k}) + \signX{X_k}\frac{A_k*\reverse{X_k}}{\norm{X_k}} \normed{X_k} \cos(\norm{X_k})\\
 &+ \frac{A- P_{X}(A)}{\norm{X}} \sin{\norm{X_k}}.
\end{split}
\end{equation}
Since $\normed{X_k}^2 = -1$, we can pull out a factor of
\begin{equation*}
\signX{X_k}\frac{A_k*\reverse{X_k}}{\norm{X_k}}\normed{X_k} = P_{X_k}(A_k)
\end{equation*}
from the trigonometric terms and rewrite that as an exponential.

The case $X_k^2 > 0$ goes identically, with the appropriate substitutions of hyperbolic functions for the ordinary ones.

The case $X_k^2 = 0$ goes as follows: first observe that $(X_k + \epsilon A_k)^2 = \epsilon (X_k A_k + A_k X_k) + \mathcal{O}(\epsilon^2)$.
Then by associativity $(X_k + \epsilon A_k)^{2n} = \mathcal{O}(\epsilon^n)$ and $(X_k + \epsilon A_k)^{2n + 1} = \mathcal{O}(\epsilon^n)$, for $n \geq 1$.
Therefore we can write the power series
\begin{equation*}
 e^{X_k + \epsilon A_k} = 1 + X_k + \epsilon A_k + \frac{1}{2} \epsilon (A_k X_k + X_k A_k) + \frac{1}{3!} \epsilon X_k A_k X_k + \mathcal{O}(\epsilon^2),
\end{equation*}
and the claim follows immediately from the definition $A_k * \partial e^{X_k} = \frac{\dif}{\dif \epsilon} e^{X_k + \epsilon A_k} \vert_{\epsilon = 0}$.

\printbibliography[heading=bibintoc, title={References}]

\end{document}
