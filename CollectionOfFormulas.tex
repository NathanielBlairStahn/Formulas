\documentclass[a4paper,12pt]{article}
\usepackage{amsmath, amsthm}
\usepackage{amsfonts}
\usepackage{amssymb}
\usepackage[utf8]{inputenc}
\usepackage[T1]{fontenc}
\usepackage{slashed}
\usepackage{commath}
\usepackage[style=phys, biblabel=brackets, eprint=true, sorting=none, backend=biber, maxnames=99]{biblatex}
\usepackage{graphicx}

%Bibliographies
\addbibresource{References.bib}

%Page Settings
\setlength{\textwidth}{14.5cm}
\setlength{\textheight}{22.5cm}
\setlength{\topmargin}{-5mm}%{-1.5cm}
%\setlength{\evensidemargin}{0.36cm}
\setlength{\oddsidemargin}{5mm}%{0.36cm}

%Better, emptier empty set
\renewcommand{\emptyset}{\varnothing}

\newcommand{\GA}[2][]{\ensuremath{\mathcal{G}_{#1}({#2})}}
\newcommand{\grade}[2]{\ensuremath{\langle#2\rangle_{#1}}}
\newcommand{\deriv}[2]{\ensuremath{\frac{\mathrm{d}{#1}}{\mathrm{d}{#2}}}}
\newcommand{\cev}[1]{\reflectbox{\ensuremath{\vec{\reflectbox{\ensuremath{#1}}}}}}
\newcommand{\reverse}[1]{\tilde{#1}} %The Clifford reverse
\newcommand{\reals}[1]{\ensuremath{\mathbb{R}^{#1}}}
\newcommand{\vol}[1]{\ensuremath{\mathrm{vol}(#1)}}
\newcommand{\linner}{\mathbin{\ensuremath{\scalebox{1.4}{$\lrcorner$}}}}
\newcommand{\rinner}{\mathbin{\ensuremath{\scalebox{1.4}{$\llcorner$}}}}

\newcommand{\comment}[1]{}

\providecommand{\abs}[1]{\lvert#1\rvert}
\providecommand{\norm}[1]{\left\lVert#1\right\rVert}
\providecommand{\normed}[1]{\hat{#1}}
\DeclareMathOperator{\sign}{sgn}
\DeclareMathOperator{\tr}{Tr}
\providecommand{\signX}[1]{\ensuremath{s_{#1}}}

\newenvironment{minipeqn}[1][]{\begin{minipage}[#1]{.45\textwidth}\begin{equation}}{\nonumber\end{equation}\end{minipage}}

%opening
\title{Formulas in Geometric Algebra and Geometric Calculus}

\author{}

\begin{document}

\maketitle

\section{Notation}

Unless otherwise stated, lower case latin alphabet stand for vectors, capital latin alphabet for multivectors, and $A_k$ for a multivector with only grade $k$ components, $A_k = \grade{k}{A_k}$.

We use both the left and right inner products and the "grade symmetric" inner product, which ever is more convenient:
\begin{eqnarray}
A_r \linner B_s &=& \grade{s-r}{A_r B_s}\\
A_r \rinner B_s &=& \grade{r-s}{A_r B_s}\\
A_r \cdot B_s &=& \grade{\abs{r-s}}{A_r B_s},\ \textrm{if}\ r, s \neq 0\\
A_r \cdot B_s &=& 0,\ \textrm{if}\ r = 0\ \textrm{or}\ s = 0\\
a \linner b &=& a \rinner b = a \cdot b
\end{eqnarray}

We define $\signX{X} := \sign{X * \reverse{X}}$ for multivector $X$.
In a space with positive definite signature $\signX{X} = 1$ for all multivectors.

Scalar product and norm:
\begin{align}
 A * B &= \grade{}{AB}\\
 \norm{A} &= \sqrt{|\reverse{A} * A|}\\
 \normed{A} &= \frac{A}{\norm{A}}
\end{align}

Inverse, when $A_k$ a $k$ -blade and $\norm{A_k} \neq 0$, \emph{i.e.} $A_k$ is an invertible blade:
\begin{equation}
\label{eq:inversedef}
A_k^{-1} = \frac{A_k}{A_k^2} = \frac{\signX{A_k}\reverse{A_k}}{\norm{A_k}^2}.
\end{equation}

Projection to subspace defined by $k$ -blade $B_k$, when $\norm{B_k} \neq 0$:
\begin{eqnarray}
\label{eq:projdef}
 P_{B_k}(A) &=& B^{-1}_{k}(B_k \rinner A)
\end{eqnarray}

\section{Geometric Algebra}

\subsection{Basic identities}

Inner, wedge and geometric products with vectors\cite{CA2GC}:
%page 8, (1.27a, b), (1.28)
\begin{eqnarray}
a\cdot A_r &=& \frac{1}{2}(a A_r - (-1)^r A_r a) = a\linner A_r\\
a\wedge A_r &=& \frac{1}{2}(a A_r + (-1)^r A_r a)\\
a A_r &=& a \cdot A_r + a \wedge A_r
\end{eqnarray}
Reordering \cite{CA2GC, Chisolm:2012aa}:
%GA2GC page 8, (1.23a, b), Chisolm page 83, (10-11) 
\begin{eqnarray}
 A_r\cdot B_s &=& (-1)^{r(s-1)} B_s \cdot A_r\ \textrm{for}\ r \leq s\\
 A_r\linner B_s &=& (-1)^{r(s-1)} B_s \rinner A_r\\
 A_r\wedge B_s &=& (-1)^{rs} B_s \wedge A_r\\
\end{eqnarray}
More generally \cite{CA2GC, Chisolm:2012aa}:
%Chisolm p. 83, (2), GA2GC p. 6, (1.20d)
\begin{eqnarray}
 \grade{r+s-2j}{A_r B_s} &=& (-1)^{rs - j} \grade{r + s - 2j}{B_s A_r}\\
 \grade{q}{A_r B_s C_t} &=& (-1)^{(q^2 + r^2 + s^2 + t^2 - q - r - s - t)/2} \grade{q}{C_t B_s A_r}
\end{eqnarray}

\subsection{Functions}

The exponential:
\begin{align}
\label{eq:expdef}
\exp(A) := e^{A} =  \sum_{k = 0}^\infty \frac{1}{k!} A^k.
\end{align}
Via the standard proof, the exponential converges in the $\norm{\cdot}$ -norm for all $A$.

For blade $A_k$,
\begin{align}
\label{eq:bladeexptrig}
e^{A_k} = \left\{\begin{aligned}
	 &\cos(\norm{A_k}) + \normed{A_k} \sin(\norm{A_k}), &\sign{A_k^2} < 0&\\
	&\cosh(\norm{A_k}) + \normed{A_k} \sinh(\norm{A_k}), &\sign{A_k^2} > 0&\\
	&1 + A_k, &A_k^2 = 0&
	\end{aligned} \right.
\end{align}

\section{Directed derivatives}

The variable being differentiated is denoted $X$ when the formula is valid for multivectors, and $x$ when it applies only to vectors. Correspondingly, the direction being differentiated in is denoted $a$ and $A$ for vector and multivector directions, respectively.

We assume that the direction $A$ only contains the grades in $X$. If not, the $A$ on the right hand side becomes $P(A)$, where $P$ is the projection to the grades of $X$.

Note that we use the definition $A * B = \grade{}{A B}$ for the scalar product. Changing this would give slight differences in the results.

Elementary derivatives \cite{HitzerCalculus}:
\begin{eqnarray}
%Hitzer page 6, 7
\label{eq:iddiff}
A * \partial  X &=& A\\
\label{eq:scalpdiff}
A * \partial (X * B)  &=&  A * B
\end{eqnarray}

Given a multivector valued functions $f(X)$ and $F(X)$, the function $G(X) := F(f(X))$ obeys the chain rule \cite{HitzerCalculus}:
\begin{equation}
%Hitzer page 7:
\label{eq:chainrule}
A * \partial G(X) = (A * \partial f(X))*\partial_Y F(Y)\vert_{Y = f(X)}.
\end{equation}

The norm and the unit multivector, for $\norm{X} \neq 0$:
\begin{align}
%proofs document
\label{eq:normsqrdif}
A * \partial \norm{X}^2 &= \signX{X}2 A * \reverse{X}\\
\label{eq:normdif}
A * \partial \norm{X} &= \signX{X} \frac{A * \reverse{X}}{\norm{X}}\\
\label{eq:normeddif}
A * \partial \normed{X} &= \frac{A \norm{X} - \signX{X} A * \normed{X} \reverse{X}}{\norm{X}^2}\quad \textrm {for all $X$}\nonumber\\
&= \frac{A - A * X X^{-1}}{\norm{X}} = \frac{A- P_{X}(A)}{\norm{X}}\quad \textrm{when $X$ an invertible blade}\\
\label{eq:normtokdif}
A * \partial \norm{X}^k &= \signX{X} k A * \normed{\reverse{X}} \norm{X}^{k-1}
\end{align}

Powers of a $k$-blade $X_k$ in the direction of $k$-vector $A_k$:
%Should we come up with a notation for a blade, or is the distinction $X_k$ enough?
\begin{align}
%proofs document
\label{eq:evenbladedif}
A_k * \partial X_k^n &= n A_k *X_k X_k^{n - 2} \quad n\ \mathrm{even}&&\\
%&= n  \signX{X_k}^{\frac{n}{2} - 1}(-1)^{\frac{k(k-1)}{2}(\frac{n}{2} - 1)}A_k * \normed{X_k} \norm{X_k}^{n-1}\\
&= n A_k * X_k^{-1} X_k^{n},\quad n \textrm{ even and invertible}&&\\
A_k * \partial X_k^n &= A_k X_k^{n-1} + (n-1) A_k*X_k X_k^{n-2}
\quad n \textrm{ odd}&&\nonumber\\
\label{eq:oddbladedif}
&= \left(A_k + (n-1) P_{X_k}(A_k)\right) X_k^{n-1},\quad n\ \textrm{odd and invertible}&&
\end{align}

Exponential of a blade:
\begin{equation}
%proofs document
\label{eq:bladeexpdif}
A_k *\partial e^{X_k} = \left\{\begin{aligned}
&P_{X_k}(A_k) e^{X_k} + (A_k - P_{X_k}(A_k)) \frac{\sin(\norm{X_k})}{\norm{X_k}}, &\sign{X_k^2} < 0&\\
&P_{X_k}(A_k) e^{X_k} + (A_k - P_{X_k}(A_k)) \frac{\sinh(\norm{X_k})}{\norm{X_k}}, &\sign{X_k^2} > 0&\\
&A_k + \frac{1}{2}(A_k X_k + X_k A_k) + \frac{1}{6}X_k A_k X_k, &X_k^2 = 0&
\end{aligned} \right.
\end{equation}

\section{Vector derivatives}

\section{Antiderivatives}


\printbibliography[heading=bibintoc, title={References}]

\end{document}
