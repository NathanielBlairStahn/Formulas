\documentclass[a4paper,12pt]{article}
\usepackage{amsmath, amsthm}
\usepackage{amsfonts}
\usepackage{amssymb}
\usepackage[utf8]{inputenc}
\usepackage[T1]{fontenc}
\usepackage{slashed}
\usepackage{commath}
\usepackage[style=phys, biblabel=brackets, eprint=true, sorting=none, backend=biber, maxnames=99]{biblatex}
\usepackage{graphicx}

%Bibliographies
\addbibresource{References.bib}

%Page Settings
\setlength{\textwidth}{14.5cm}
\setlength{\textheight}{22.5cm}
\setlength{\topmargin}{-5mm}%{-1.5cm}
%\setlength{\evensidemargin}{0.36cm}
\setlength{\oddsidemargin}{5mm}%{0.36cm}

%Better, emptier empty set
\renewcommand{\emptyset}{\varnothing}

\newcommand{\GA}[2][]{\ensuremath{\mathcal{G}_{#1}({#2})}}
\newcommand{\grade}[2]{\ensuremath{\langle#2\rangle_{#1}}}
\newcommand{\deriv}[2]{\ensuremath{\frac{\mathrm{d}{#1}}{\mathrm{d}{#2}}}}
\newcommand{\cev}[1]{\reflectbox{\ensuremath{\vec{\reflectbox{\ensuremath{#1}}}}}}
\newcommand{\reverse}[1]{\tilde{#1}} %The Clifford reverse
\newcommand{\reals}[1]{\ensuremath{\mathbb{R}^{#1}}}
\newcommand{\vol}[1]{\ensuremath{\mathrm{vol}(#1)}}
\newcommand{\linner}{\ensuremath{\rfloor}}
\newcommand{\rinner}{\ensuremath{\lfloor}}

\newcommand{\comment}[1]{}

\providecommand{\abs}[1]{\lvert#1\rvert}
\providecommand{\norm}[1]{\left\lVert#1\right\rVert}
\DeclareMathOperator{\sign}{sgn}
\DeclareMathOperator{\tr}{Tr}

\newenvironment{minipeqn}[1][]{\begin{minipage}[#1]{.45\textwidth}\begin{equation}}{\nonumber\end{equation}\end{minipage}}

%opening
\title{Formulas in Geometric Algebra and Geometric Calculus}

\author{}

\begin{document}

\maketitle

\section{Notation}

Unless otherwise stated, lower case latin alphabet stand for vectors, capital latin alphabet for multivectors, and $A_k$ for a multivector with only grade $k$ components, $A_k = \grade{k}{A_k}$.

We use both the left and right inner products and the "grade symmetric" inner product, which ever is more convenient:
\begin{eqnarray}
A_r \linner B_s &=& \grade{s-r}{A_r B_s}\\
A_r \rinner B_s &=& \grade{r-s}{A_r B_s}\\
A_r \cdot B_s &=& \grade{\abs{r-s}}{A_r B_s},\ \textrm{if}\ r, s \neq 0\\
A_r \cdot B_s &=& 0,\ \textrm{if}\ r = 0\ \textrm{or}\ s = 0\\
a \linner b &=& a \rinner b = a \cdot b
\end{eqnarray}

\section{Geometric Algebra}

\subsection{Basic identities}

Inner, wedge and geometric products with vectors\cite{CA2GC}:
\begin{eqnarray}
a\cdot A_r &=& \frac{1}{2}(a A_r - (-1)^r A_r a) = a\linner A_r\\
a\wedge A_r &=& \frac{1}{2}(a A_r + (-1)^r A_r a)\\
a A_r &=& a \cdot A_r + a \wedge A_r
\end{eqnarray}
Reordering \cite{CA2GC, Chisolm:2012aa}:
\begin{eqnarray}
 A_r\cdot B_s &=& (-1)^{r(s-1)} B_s \cdot A_r\ \textrm{for}\ r \leq s\\
 A_r\linner B_s &=& (-1)^{r(s-1)} B_s \rinner A_r\\
 A_r\wedge B_s &=& (-1)^{rs} B_s \rinner A_r\\
\end{eqnarray}
More generally \cite{CA2GC, Chisolm:2012aa}:
\begin{eqnarray}
 \grade{r+s-2j}{A_r B_s} &=& (-1)^{rs - j} \grade{r + s - 2j}{B_s A_r}\\
 \grade{q}{A_r B_s C_t} &=& (-1)^{(q^2 + r^2 + s^2 + t^2 - q - r - s - t)/2} \grade{q}{C_t B_s A_r}
\end{eqnarray}
\section{Directed derivatives}

\section{Vector derivatives}

\section{Antiderivatives}


\printbibliography[heading=bibintoc, title={References}]

\end{document}
